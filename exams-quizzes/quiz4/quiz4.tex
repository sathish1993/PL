\documentclass[12pt]{article}
%\usepackage{html}
\usepackage{hyperref}
\usepackage{graphicx}
\title{CS 571\\Quiz 4}
%\addtolength{\topmargin}{-2cm}
%\addtolength{\topskip}{-2cm}
%\addtolength{\oddsidemargin}{-2cm}
%\addtolength{\evensidemargin}{-2cm}
%\addtolength{\textheight}{2cm}
%\addtolength{\textwidth}{2cm}
%\addtolength{\footskip}{-1cm}
\date{}
\begin{document}
\maketitle

\begin{flushleft}
\textbf{Nov 9}\hfill\textbf{Closed book}\\
\textbf{15 points}\hfill\textbf{Closed notes}\\

\vspace{0.5cm}

\textbf{Important Reminder}: As per the course Academic Honesty
Statement, cheating of any kind will minimally result in receiving an
F letter grade for the entire course.


\end{flushleft}

\textbf{Please ensure that you have filled-in BOTH your name and
  B-number in the bubbles on the provided grid-sheet.}

For each of the following questions, select a \textbf{single}
alternative on the grid-sheet.  

There are 7 questions with 2-points per question; there is 1-point
for submitting the quiz.

\begin{enumerate}

\item What will be the result of evaluating the following Haskell expression?

\begin{verbatim}
length [[1, 2], [3, 4], [5]]
\end{verbatim}

\begin{enumerate}

\item An error will occur.
\item \verb@2@.  
\item \verb@3@.  
\item \verb@4@.  
\item \verb@5@.  
  
\end{enumerate}

\newpage

\item What will be the result of evaluating the following Haskell expression?

\begin{verbatim}
length [(x, y) | x <- [1..10], y <- "abcde"]  
\end{verbatim}

\begin{enumerate}

\item An error will occur.

\item 10.

\item 15.

\item 20.

\item 50.  

\end{enumerate}


\item What will be the result of evaluating the following Haskell expression?

\begin{verbatim}
foldl (-) 1 [1, 2, 3]
\end{verbatim}

\begin{enumerate}
\item An error will occur.

\item \verb@1@.

\item \verb@-1@
  
\item \verb@5@.

\item \verb@-5@.

  
\end{enumerate}


\item What will be the value of evaluating the following Haskell
  expression?

\begin{verbatim}
foldr (-) 1 [1, 2, 3]
\end{verbatim}

\begin{enumerate}

\item An error will occur.

\item \verb@1@.

\item \verb@-1@
  
\item \verb@5@.

\item \verb@-5@.

\end{enumerate}

\newpage

\item Which of the following is not a legal Haskell expression?

\begin{enumerate}

\item \verb@[1, 2, 3]@.

\item \verb@['a', 'b'] ++ "c"@.

\item \verb@[1, 2, [1]]@.

\item \verb@[['a', 'b'], "cd"]@.

\item \verb@[[1], [2]]@.

\end{enumerate}


\item What will be the result of evaluating the following Haskell expression?

\begin{verbatim}
foldr (++) "x" ["abc", "de", "f"] 
\end{verbatim}

\begin{enumerate}

\item An error will occur.

\item \verb@"abcdef"@.

\item \verb@"fedcba"@.  
  
\item \verb@"xabcdef"@.

\item \verb@"abcdefx"@.

\end{enumerate}


\item Given sets $A = \{1, 2, 3, 4\}$ and $B = \{a, b, c, d\}$, which of
  the following is \textbf{not} a function from $A$ to $B$.

\begin{enumerate}
\item $\{\}$.

\item $\{(1, a), (2, a), (3, a), (4, a)\}$.  

\item $\{(1, a), (2, b), (3, c), (4, d)\}$.  

\item $\{(1, b), (2, b), (1, d), (4, d)\}$.  

\item $\{(1, a)\}$.  

  
\end{enumerate}

\end{enumerate}

\end{document}
