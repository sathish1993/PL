\documentclass[12pt]{article}
%\usepackage{html}
\usepackage{hyperref}
\usepackage{graphicx}
\title{CS 571\\Quiz 3}
%\addtolength{\topmargin}{-2cm}
%\addtolength{\topskip}{-2cm}
%\addtolength{\oddsidemargin}{-2cm}
%\addtolength{\evensidemargin}{-2cm}
%\addtolength{\textheight}{2cm}
%\addtolength{\textwidth}{2cm}
%\addtolength{\footskip}{-1cm}
\date{}
\begin{document}
\maketitle

\begin{flushleft}
\textbf{Oct 10}\hfill\textbf{Closed book}\\
\textbf{15 points}\hfill\textbf{Closed notes}\\

\vspace{0.5cm}

\textbf{Important Reminder}: As per the course Academic Honesty
Statement, cheating of any kind will minimally result in receiving an
F letter grade for the entire course.


\end{flushleft}

\textbf{Please ensure that you have filled-in BOTH your name and
  B-number in the bubbles on the provided grid-sheet.}

For each of the following questions, select a \textbf{single}
alternative on the grid-sheet.  

There are 7 questions with 2-points per question; there is 1-point
for submitting the quiz.

\begin{enumerate}

\item Which of the following languages over the vocabulary of
  square-brackets $\{$\verb@[@, \verb@]@$\}$ is not expressible using
  standard regular expressions?

\begin{enumerate}

\item Strings of even length.  Examples include
  the empty string, \verb@][@, \verb@[[[]@ and \verb@[[]]@.

\item Strings of even length which consist of balanced brackets.
  Examples include the empty string, \verb@[]@ and \verb@[][[]]@.

\item Strings of even length containing 2-or-more \verb@]@'s
  followed by 0-or-more \verb@[@'s.  Examples include \verb@]]@,
  \verb@]]][@ and \verb@]][[@.

\item Strings of length less-than-or-equal-to 4 which consist of
  balanced brackets. 
    Examples include the empty string, \verb@[]@ and \verb@[][]@.

\item Strings whose length is exactly 4.  Examples include \verb@]][[@,
  \verb@[[[]@ and \verb@[][]@.
  
\end{enumerate}

\item In Javascript, \textit{hoisting} refers to:
\begin{enumerate}

\item Moving \verb@let@ declarations to the start
  of a block.

\item Moving \verb@let@ declarations to the start
  of a function.

\item Moving \verb@var@ declarations to the start
  of a block.

\item Moving \verb@var@ declarations to the start
  of a function.

\item Moving \verb@var@ declarations to the \verb@window@ object.

\end{enumerate}

\item What should be the value of the following Scheme expression?

\begin{verbatim}

    (length '( 1 (2 3) () () (()) ))

 \end{verbatim}


\begin{enumerate}

\item It will result in an error since the list is not a proper list.
  
\item 3

\item 4 
  
\item 5
  
\item 6
    
\end{enumerate}

\newpage




\item Which of the following is the most accurate characterization of
 the semantics of \verb@cons@, \verb@car@ and \verb@cdr@ in
 Scheme?

\begin{enumerate}

\item \verb@cons@ constructs a list, \verb@car@ returns the head
of the list, \verb@cdr@ returns the tail of the list.

\item \verb@cons@ constructs a list, \verb@car@ returns the tail
of the list, \verb@cdr@ returns the head of the list.

\item \verb@cons@ constructs a pair, \verb@car@ returns the first
element of the pair, \verb@cdr@ returns the second element of the pair.

\item \verb@cons@ constructs a list, \verb@car@ returns the first
 element of the list and \verb@cdr@ returns the second element of the
 list.

\item \verb@cons@ constructs a pair, \verb@car@ returns the second
element of the pair, \verb@cdr@ returns the first element of the pair.


\end{enumerate}

\item What should be the value of the following Scheme expression?

\begin{verbatim}

      (caddr '(a b c d e))

\end{verbatim}

\begin{enumerate}

  \item \verb@'b@.

  \item \verb@'c@.
    
  \item \verb@'d@.

  \item \verb@'(c d e)@

  \item \verb@'(d e)@
    
    
\end{enumerate}

\newpage

\item What should be the value of the following Scheme expression?

\begin{verbatim}

      (cdddr '(a b c d e))

\end{verbatim}

\begin{enumerate}

  \item \verb@'b@.

  \item \verb@'c@.
    
  \item \verb@'d@.

  \item \verb@'(c d e)@

  \item \verb@'(d e)@
    
\end{enumerate}
    
\newpage



\item Given the following tree structure:

  \begin{center}
    \includegraphics[height=80mm]{nest1}\\
  \end{center}

  which of the following Scheme expressions best describes the
  structure?

\begin{enumerate}

\item \verb@'(a b c d)@

\item \verb@'(a (b c) d)@

\item \verb@'(a (b . c) d)@
  
\item \verb@'(a (b c) (d ()))@


\item \verb@'(a (b c) (d . ()))@

\end{enumerate}

\end{enumerate}

\end{document}
