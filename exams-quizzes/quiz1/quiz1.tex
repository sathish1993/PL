\documentclass[12pt]{article}
%\usepackage{html}
\usepackage{hyperref}
\title{CS 571\\Quiz 1}
%\addtolength{\topmargin}{-2cm}
%\addtolength{\topskip}{-2cm}
%\addtolength{\oddsidemargin}{-2cm}
%\addtolength{\evensidemargin}{-2cm}
%\addtolength{\textheight}{2cm}
%\addtolength{\textwidth}{2cm}
%\addtolength{\footskip}{-1cm}
\date{}
\begin{document}
\maketitle

\begin{flushleft}
\textbf{Sep 19}\hfill\textbf{Closed book}\\
\textbf{15 points}\hfill\textbf{Closed notes}\\

\vspace{0.5cm}

\textbf{Important Reminder}: As per the course Academic Honesty
Statement, cheating of any kind will minimally result in receiving an
F letter grade for the entire course.


\end{flushleft}

For each of the following questions, select a \textbf{single}
alternative on the grid-sheet.  Please ensure that you have filled-in
your name in the bubbles on the provided grid-sheet; the B-number is
not required.

There are 7 questions with 2-points per question; there is 1-point
for submitting the quiz.

\begin{enumerate}

\item Which of the following statements about standard regular
  expression notation is false?

\begin{enumerate}

\item The infix \verb@.@ binary operator is used to denote the concatenation
  of two regular expressions.

\item The infix \verb@|@ binary operator is used to denote the alternation
  of two regular expressions.

\item The postfix \verb@*@ unary operator is used to denote 0-or-more
  repetitions of a regular expression.

\item The postfix \verb@?@ unary operator is used to denote an
optional regular expression.

\item The postfix \verb@+@ unary operator is used to denote 1-or-more
  repetitions of a regular expression.
    
\end{enumerate}

\newpage

\item Which of the following programming languages is very different
  from the others in terms of syntax?
\begin{enumerate}

\item Algol

\item Pascal

\item Lisp

\item C

\item C++

\end{enumerate}


\item Which of the following regular expressions describes strings of
  one-or-more \verb@a@'s followed \textbf{optionally} by a single
  \verb@b@ followed by zero-or-more \verb@a@'s (ignore whitespace added
  for readability within each regex)?

\begin{enumerate}

\item \verb@a? b* a+@

\item \verb@a a* b? a*@

\item \verb@a* b? a* a@

\item \verb@a+ b? a+@

\item \verb@a* b+ a+@

\end{enumerate}


\item Which of the following languages over the vocabulary $\{$\verb@a@, \verb@b@$\}$ is not expressible using standard regular expressions?

\begin{enumerate}

\item Strings whose length is exactly 5.  Examples include \verb@aabab@,
  \verb@babab@ and \verb@aaaaa@.
  
\item Strings whose length must be a multiple of 3.  Examples include
  the empty string, \verb@aba@, \verb@aabbab@.

\item Strings of length less-than-or-equal-to 8 which consist of
  a sequence of \verb@a@'s followed by an equal number of \verb@b@'s.
  Examples include the empty string, \verb@ab@ and \verb@aaabbb@.

\item Strings of arbitrary length containing 1-or-more \verb@a@'s
  followed by 0-or-more \verb@b@'s.  Examples include \verb@aaa@,
  \verb@a@ and \verb@abbb@.

\item Strings of arbitrary length which consist of
  a sequence of \verb@a@'s followed by an equal number of \verb@b@'s.
  Examples include the empty string, \verb@ab@ and \verb@aaaabbbb@.

\end{enumerate}

\newpage

\item Given the following CFG over the set of terminal symbols 
  \verb@NUMBER@, \verb@#@, \verb@^@, \verb@!@, \verb@(@ and \verb@)@:

\begin{verbatim}

exp
  : exp '#' term
  | term
  ;
term
  : factor '^' term
  | factor
  ;
factor
  : factor '!'
  | '(' exp ')'
  | NUMBER
  ;
\end{verbatim}

Which of the following statements about the precedence and associativity of
the operators \verb@#@, \verb@^@, and \verb@!@ is true?

\begin{enumerate}

  \item \verb@^@ has lowest precedence, followed by \verb@#@ with
    higher precedence, followed by \verb@!@ with highest precedence.
    \verb@#@ is right-associative, while \verb@^@ is left-associative.

  \item \verb@#@ has lowest precedence, followed by \verb@^@ with
    higher precedence, followed by \verb@!@ with highest precedence.
    \verb@#@ is left-associative, while \verb@^@ is right-associative.

  \item \verb@!@ has lowest precedence, followed by \verb@^@ with
    higher precedence, followed by \verb@#@ with highest precedence.
    \verb@#@ is left-associative, while \verb@^@ is right-associative.

  \item \verb@#@ has lowest precedence, followed by \verb@^@ with
    higher precedence, followed by \verb@!@ with highest precedence.
    \verb@#@ is right-associative, while \verb@^@ is left-associative.

  \item \verb@!@ has lowest precedence, followed by \verb@^@ with
    higher precedence, followed by \verb@#@ with highest precedence.
    \verb@#@ is right-associative, while \verb@^@ is left-associative.

      
    
\end{enumerate}

\newpage

\item Which of the following describes the language consisting of
$n$ \verb@a@'s followed by exactly $n$ \verb@b@'s for $n >= 0$?

\begin{enumerate}

\item The regular expression \verb@a*b*@.

\item The CFG:
\begin{verbatim}
           S
             : 'a' S 'b'
             | //empty
             ;
\end{verbatim}

\item The regular expression \verb@a+b+@.

\item The CFG:
\begin{verbatim}
           S
             : 'a' S 'b'
             | 'a' 'b'
             ;
\end{verbatim}

\item The CFG:
\begin{verbatim}
           S
             : 'a' 'a' S 'b' 'b'
             | //empty
             ;
\end{verbatim}

\end{enumerate}

\newpage

\item Which of the following statements is false?

\begin{enumerate}

\item A \textit{scanner} converts a stream of characters into a stream of tokens.

\item A recursive-descent parser must have a parsing function for each
  non-terminal in the grammar.

\item The stack frame for a function activation will typically contain
  the return address for that activation.

\item If a grammar permits a derivation containing a step with
  ambiguity about which non-terminal should be expanded next, then
  the grammar is defined to be \textit{ambiguous}.

\item The \verb@match()@ function of a recursive-descent parser must
match the current terminal, else signal an error.

\end{enumerate}

\end{enumerate}

\end{document}
