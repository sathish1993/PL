\documentclass[12pt]{article}
%\usepackage{html}
\usepackage{hyperref}
\usepackage{graphicx}
\title{CS 571\\Midterm}
%\addtolength{\topmargin}{-2cm}
%\addtolength{\topskip}{-2cm}
%\addtolength{\oddsidemargin}{-2cm}
%\addtolength{\evensidemargin}{-2cm}
%\addtolength{\textheight}{2cm}
%\addtolength{\textwidth}{2cm}
%\addtolength{\footskip}{-1cm}
\date{}
\begin{document}
\maketitle

\begin{flushleft}
\textbf{Oct 24}\\
\textbf{100 points}\hfill\textbf{Open book, open notes}\\
\textbf{Time}: 85 minutes\hfill\textbf{No Electronic Devices}\\

\vspace{0.5cm}

\textbf{Important Reminder}: As per the course Academic Honesty
Statement, cheating of any kind will minimally result in receiving an
F letter grade for the entire course.

\textbf{Justify all answers}

Please write your answers only within the provided exam booklets.

\end{flushleft}

There are a total of 6 questions.

\begin{enumerate}

\item A Unix \textit{path} is defined in stages as follows:

  \begin{description}
  \item[\textbf{Path Component}]
    A \textit{path component} is a sequence of one-or-more
  characters which does not contain any occurrences of the \verb@/@ or
  \verb@NUL@ characters.
\item[\textbf{Relative Path}] A \textit{relative path} is a sequence
  of one-or-more path components separated by a single \verb@/@
  character.
\item[\textbf{Absolute Path}]
  An \textit{absolute path} consists of the \verb@/@ character
  optionally followed by a relative path.
\item[\textbf{Unix Path}]
  A Unix path is either an absolute or a relative path.
    \end{description}

  Provide a regex for Unix paths.  You may use \verb@\/@ to represent
  the regex matching \verb@/@ and \verb@\0@ to represent the regex
  matching the \verb@NUL@ character.  You should factor (using
  intermediate named regex's) or format your answer to ensure that
  it is readable and understandable.
  \hfill{\textit{10-points}}
  

\item An \textit{X-expression} is either an atom, or two X-expressions
  surrounded by parentheses and separated by \verb@.@ (period), or a
  sequence of one-or-more X-expressions surrounded by parentheses.

  \begin{enumerate}
  \item Give a grammar for \textit{X-expressions}.  You should
    use the set of terminals $\{$ \verb@ATOM@$,$ \verb@'('@$,$ \verb@')'@$,$
      \verb@'.'@$\}$.

  \item Use your grammar to provide a \textit{parse tree} for the
    X-expression \verb@( (1 . 2) 3)@, where the integers will be
    scanned as \verb@ATOM@ terminals.

  \end{enumerate} \hfill\textit{20-points}
  
\item Given the following program in a language which supports nested
  functions as well as both lexically-scoped (indicated using a \verb@lex@
  declaration) and dynamically-scoped variables (indicated using a
  \verb@dyn@ declaration):

\begin{verbatim}
lex lex1 = 1;  //lexically scoped var lex1
dyn dyn1 = 2;  //dynamically scoped var dyn1

f(param_f) { //define function f with single parameter param_f
  lex lex1 = 3; 
  dyn dyn1 = 4;

  g(param_g) { //define function g with single parameter param_g

    return lambda(x) { return x + param_f*param_g + lex1*dyn1; };
  }

  return g(dyn1);

}

print f(6)(7);
  
\end{verbatim}


What will be printed by the above program.  Please remember to justify
your answer.  \hfill\textit{20-points}

\item Describe how you would represent a CFG using basic
  S-expressions.

\begin{enumerate}  
  \item Specifically, describe how you would use
  S-expressions to represent \textit{terminals},
  \textit{non-terminals}, \textit{rules} and \textit{grammars}.

  \item Show your representation for the example CFG:
\begin{verbatim}
s : a
  | b
  ;
a : A
  ;
b : B
  ;
\end{verbatim}

  \item Describe how you would hide the details of your representation from
    users of your representation.

\end{enumerate}
\hfill\textit{15-points}

\item Write a Scheme function
  \verb@(count-atoms @\textit{s-exp}\verb@)@ which counts the number
  of atoms (non-pairs) in a maximally simplified representation of
  S-expression \textit{s-exp} (i.e., the \verb@'()@ terminating proper
  lists should be ignored).

Example log:
\begin{verbatim}
> (count-atoms 'a)
1
> (count-atoms '(a ()))
2
> (count-atoms '(a . ()))
1
> (count-atoms '(a b . ()))
2
> (count-atoms '(a b ()))
3
>
\end{verbatim}  \hfill\textit{20-points}

\newpage

\item Discuss the validity of the following statements:

  \begin{enumerate}

  \item All evaluable Scheme expressions are S-expressions.

  \item All S-expressions are evaluable Scheme expressions.

  \item Assuming that a stack grows towards high memory, then within a
    stack frame for a function, the parameters to the function will be
    located at higher addresses than the local variables of the
    function.

  \item Languages which allow recursive functions \textbf{must} use
    stack allocation for function parameters and local variables.

  \item If a language requires a left-associative binary operator $\oplus$
    to have the same precedence as a right-associative binary operator
    $\otimes$, then the language is ambiguous.

    \end{enumerate}
\hfill\textit{15-points}

\end{enumerate}

\end{document}
