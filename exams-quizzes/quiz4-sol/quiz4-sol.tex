\documentclass[12pt]{article}
%\usepackage{html}
\usepackage{hyperref}
\usepackage{graphicx}
\title{CS 571\\Quiz 4 Solution}
%\addtolength{\topmargin}{-2cm}
%\addtolength{\topskip}{-2cm}
%\addtolength{\oddsidemargin}{-2cm}
%\addtolength{\evensidemargin}{-2cm}
%\addtolength{\textheight}{2cm}
%\addtolength{\textwidth}{2cm}
%\addtolength{\footskip}{-1cm}
\date{}
\begin{document}
\maketitle

\begin{flushleft}
  \textbf{Nov 9}   (Actually held on Nov 14)\hfill\textbf{Closed book}\\
\textbf{15 points}\hfill\textbf{Closed notes}\\

\vspace{0.5cm}

\textbf{Important Reminder}: As per the course Academic Honesty
Statement, cheating of any kind will minimally result in receiving an
F letter grade for the entire course.


\end{flushleft}

\textbf{Please ensure that you have filled-in BOTH your name and
  B-number in the bubbles on the provided grid-sheet.}

For each of the following questions, select a \textbf{single}
alternative on the grid-sheet.  

There are 7 questions with 2-points per question; there is 1-point
for submitting the quiz.

\begin{enumerate}

\item What will be the result of evaluating the following Haskell expression?

\begin{verbatim}
length [[1, 2], [3, 4], [5]]
\end{verbatim}

\begin{enumerate}

\item An error will occur.
\item \verb@2@.  
\item \verb@3@.  
\item \verb@4@.  
\item \verb@5@.  
  
\end{enumerate}

\textbf{Answer}: (c).

\verb@length@ returns the number of elements in a list.  In this case,
the argument list has 3 elements (which themselves are lists).  Hence
the \verb@length@ will evaluate as 3.

\item What will be the result of evaluating the following Haskell expression?

\begin{verbatim}
length [(x, y) | x <- [1..10], y <- "abcde"]  
\end{verbatim}

\begin{enumerate}

\item An error will occur.

\item 10.

\item 15.

\item 20.

\item 50.  

\end{enumerate}

\textbf{Answer}: (e).

The list comprehension builds a list of pairs \verb@(x, y)@ with
\verb@x@ $\in$ \verb@[1, 2, 3, ..., 10]@ and \verb@y@ $\in$
\verb@['a', 'b', 'c', 'd', 'e']@.  So there are $10$ possibilities for
\verb@x@ and $5$ possibilities for \verb@y@; hence there are a
total of $10\times 5 = 50$ possible pairs.  Hence the length of
the list of pairs will be 50.

\item What will be the result of evaluating the following Haskell expression?

\begin{verbatim}
foldl (-) 1 [1, 2, 3]
\end{verbatim}

\begin{enumerate}
\item An error will occur.

\item \verb@1@.

\item \verb@-1@
  
\item \verb@5@.

\item \verb@-5@.

  
\end{enumerate}

\textbf{Answer}: (e).

The expression will apply \verb@-@ from the left with initial
value \verb@1@.  Hence the expression is equivalent to
\verb@((1-1)-2)-3@ which is -5.


\item What will be the value of evaluating the following Haskell
  expression?

\begin{verbatim}
foldr (-) 1 [1, 2, 3]
\end{verbatim}

\begin{enumerate}

\item An error will occur.

\item \verb@1@.

\item \verb@-1@
  
\item \verb@5@.

\item \verb@-5@.

\end{enumerate}

\textbf{Answer}: (b).

The expression will apply \verb@-@ from the right with initial
value \verb@1@.  Hence the expression is equivalent to
\verb@(1-(2-(3-1))@ which is 1.

\item Which of the following is not a legal Haskell expression?

\begin{enumerate}

\item \verb@[1, 2, 3]@.

\item \verb@['a', 'b'] ++ "c"@.

\item \verb@[1, 2, [1]]@.

\item \verb@[['a', 'b'], "cd"]@.

\item \verb@[[1], [2]]@.

\end{enumerate}

\textbf{Answer}: (c).

(c) cannot be typed since lists must be of homogeneous type, but the
elements of (c) are \verb@NUM@'s and a list of \verb@NUM@.  (a) is a
simple list of \verb@Num@, (b) is equivalent to \verb@"abc"@ with type
list of \verb@Char@, (d) is equivalent to \verb@["ab", "cd"]@ with
type list of list of \verb@Char@, and (e) is a list of list of
\verb@Num@.

\item What will be the result of evaluating the following Haskell expression?

\begin{verbatim}
foldr (++) "x" ["abc", "de", "f"] 
\end{verbatim}

\begin{enumerate}

\item An error will occur.

\item \verb@"abcdef"@.

\item \verb@"fedcba"@.  
  
\item \verb@"xabcdef"@.

\item \verb@"abcdefx"@.

\end{enumerate}

\textbf{Answer}: (e).

The \verb@foldr@ folds the \verb@++@ append operation over the list
and is equivalent to \verb@"abc" ++ "de" ++ "f" ++ "x"@ resulting in 
\verb@"abcdefx"@.

\item Given sets $A = \{1, 2, 3, 4\}$ and $B = \{a, b, c, d\}$, which of
  the following is \textbf{not} a function from $A$ to $B$.

\begin{enumerate}
\item $\{\}$.

\item $\{(1, a), (2, a), (3, a), (4, a)\}$.  

\item $\{(1, a), (2, b), (3, c), (4, d)\}$.  

\item $\{(1, b), (2, b), (1, d), (4, d)\}$.  

\item $\{(1, a)\}$.  

  
\end{enumerate}
\textbf{Answer}: (d).

An essential property for a function is that it must map a single
element in the domain to a unique element in the range; however, (d)
maps \verb@1@ to both \verb@b@ and \verb@d@.  The other alternatives
do not have similar violations of this function property.  Note that
(a) is a function which is undefined over all elements of its domain.

\end{enumerate}

\end{document}
