\documentclass[12pt]{article}
%\usepackage{html}
\usepackage{hyperref}
\usepackage{graphicx}
\title{CS 571\\Midterm Solution}
%\addtolength{\topmargin}{-2cm}
%\addtolength{\topskip}{-2cm}
%\addtolength{\oddsidemargin}{-2cm}
%\addtolength{\evensidemargin}{-2cm}
%\addtolength{\textheight}{2cm}
%\addtolength{\textwidth}{2cm}
%\addtolength{\footskip}{-1cm}
\date{}
\begin{document}
\maketitle

\begin{flushleft}
\textbf{Oct 24}\\
\textbf{100 points}\hfill\textbf{Open book, open notes}\\
\textbf{Time}: 85 minutes\hfill\textbf{No Electronic Devices}\\

\vspace{0.5cm}

\textbf{Important Reminder}: As per the course Academic Honesty
Statement, cheating of any kind will minimally result in receiving an
F letter grade for the entire course.

\textbf{Justify all answers}

Please write your answers only within the provided exam booklets.

\end{flushleft}

There are a total of 6 questions.

\begin{enumerate}

\item A Unix \textit{path} is defined in stages as follows:

  \begin{description}
  \item[\textbf{Path Component}]
    A \textit{path component} is a sequence of one-or-more
  characters which does not contain any occurrences of the \verb@/@ or
  \verb@NUL@ characters.
\item[\textbf{Relative Path}] A \textit{relative path} is a sequence
  of one-or-more path components separated by a single \verb@/@
  character.
\item[\textbf{Absolute Path}]
  An \textit{absolute path} consists of the \verb@/@ character
  optionally followed by a relative path.
\item[\textbf{Unix Path}]
  A Unix path is either an absolute or a relative path.
    \end{description}

  Provide a regex for Unix paths.  You may use \verb@\/@ to represent
  the regex matching \verb@/@ and \verb@\0@ to represent the regex
  matching the \verb@NUL@ character.  You should factor (using
  intermediate named regex's) or format your answer to ensure that
  it is readable and understandable.
  \hfill{\textit{10-points}}

\begin{verbatim}

path-component = [^\/\0]+
rel-path = path-component ( \/ path-component )*
abs-path = \/ rel-path?
path = abs-path | rel-path
\end{verbatim}  

Alternately, without using any intermediate named regex's:
\begin{verbatim}
\/ | \/? [^\/\0]+ (\/ [^\/\0]+)*
\end{verbatim}

\item An \textit{X-expression} is either an atom, or two X-expressions
  surrounded by parentheses and separated by \verb@.@ (period), or a
  sequence of one-or-more X-expressions surrounded by parentheses.

  \begin{enumerate}
  \item Give a grammar for \textit{X-expressions}.  You should
    use the set of terminals $\{$ \verb@ATOM@$,$ \verb@'('@$,$ \verb@')'@$,$
      \verb@'.'@$\}$.

  \item Use your grammar to provide a \textit{parse tree} for the
    X-expression \verb@( (1 . 2) 3)@, where the integers will be
    scanned as \verb@ATOM@ terminals.

  \end{enumerate} \hfill\textit{20-points}
  
  \begin{enumerate}

  \item
\begin{verbatim}
      x-expression
        : ATOM
        | '(' x-expression '.' x-expression ')'
        | '(' list ')'
        ;
      list
        : x-expression
        | x-expression list
        ;
\end{verbatim}

\newpage
\item
\begin{verbatim}

                      x-expression
                           |
                           |
                +--------- +----------+
                |          |          |
                |          |          |
               '('        list       ')'
                           |                
                           |                
                           +-------------------------+
                           |                         |
                           |                         |
                      x-expression                  list
                           |                         |          
                           |                         |                 
       +--------+----------+------ +---------+  x-expression
       |        |          |       |         |       |
       |        |          |       |         |       |
      '('  x-expression   '.' x-expression  ')'     ATOM (3)
                |                     |
                |                     |
               ATOM (1)              ATOM (2)
\end{verbatim}

  \end{enumerate}
  \item Given the following program in a language which supports nested
  functions as well as both lexically-scoped (indicated using a \verb@lex@
  declaration) and dynamically-scoped variables (indicated using a
  \verb@dyn@ declaration):

\begin{verbatim}
lex lex1 = 1;  //lexically scoped var lex1
dyn dyn1 = 2;  //dynamically scoped var dyn1

f(param_f) { //define function f with single parameter param_f
  lex lex1 = 3; 
  dyn dyn1 = 4;

  g(param_g) { //define function g with single parameter param_g

    return lambda(x) { return x + param_f*param_g + lex1*dyn1; };
  }

  return g(dyn1);

}

print f(6)(7);
  
\end{verbatim}


What will be printed by the above program.  Please remember to justify
your answer.  \hfill\textit{20-points}

When the anonymous function is defined, the parameters and lexically
scoped variables are bound. So at the point of definition of the
anonymous function, we have \verb@param_f@ with value 6,
\verb@param_g@ with value 4 (the value of \verb@dyn1@ when
\verb@g()@ is called), and \verb@lex1@ with value 3; hence the function
returned is:

\begin{verbatim}
      lambda(x) { return x + 6*4 + 3*dyn1; }
\end{verbatim}

When the anonymous function is run, the parameter \verb@x@ is bound to
7 and \verb@dyn1@ is bound to its current dynamic value 2.  Hence the
function returns \verb@7 + 6*4 + 3*2@ which is 37.  Hence 37 will be
printed.

\item Describe how you would represent a CFG using basic
  S-expressions.

\begin{enumerate}  
  \item Specifically, describe how you would use
  S-expressions to represent \textit{terminals},
  \textit{non-terminals}, \textit{rules} and \textit{grammars}.

  \item Show your representation for the example CFG:
\begin{verbatim}
s : a
  | b
  ;
a : A
  ;
b : B
  ;
\end{verbatim}

  \item Describe how you would hide the details of your representation from
    users of your representation.

\end{enumerate}
\hfill\textit{15-points}

\begin{enumerate}

\item One possible representation:

  \begin{description}
  \item[\textbf{Terminal Symbol}] A terminal symbol $T$ would be represented
    as the pair \verb@(term .@ $T$ \verb@)@ for some atom $T$. 
  \item[\textbf{Non-Terminal Symbol}] A non-terminal symbol $N$ would be represented
    as the pair \verb@(non-term .@ $N$ \verb@)@ for some atom $N$.
  \item[\textbf{Rule}] A rule would be represented by a pair where
    the first element of the pair is a non-terminal symbol representing the
    LHS of the rule and the second element of the pair is a list
    containing the symbols on the RHS of the rule.

  \item[\textbf{Grammar}] A grammar would be represented by a pair
    where the first element of the pair is a non-terminal representing
    the start-symbol of the grammar and the second element of the pair would
    be a list of the rules for the grammar.
  \end{description}

\item
\begin{verbatim}
       ((non-term . s)
        ((non-term . s) (non-term . a))
        ((non-term . s) (non-term . b))
        ((non-term . a) (term . A))
        ((non-term . b) (term . B)))  
\end{verbatim}
Note the simplification above by having the second element of both grammars
and rules be lists.

\item The representation would be hidden between a set of accessor
  functions.  Using Scheme notation, we could have something like the
  following:

\begin{verbatim}

;;return the symbol representing the start-symbol for the grammar
(g-start-symbol grammar)

;;return a list of rules for non-terminal symbol non-term in grammar.
(g-rules grammar non-term)

;;return the non-terminal symbol for rule.
(g-lhs rule)
  
;;return a list of symbols representing the rhs of rule.
(g-rhs rule)

;;Return #f iff sym does not represent a non-terminal symbol  
;;(non-terminal? sym)  
  
;;Return #f iff sym does not represent a terminal symbol  
;;(terminal? sym)  
  
\end{verbatim}

This would allow hiding the representation.  So the representation
provided in part (a) could be changed to a more efficient indexed
representation, but the use of the representation would remain
unchanged. 

\end{enumerate}

\item Write a Scheme function
  \verb@(count-atoms @\textit{s-exp}\verb@)@ which counts the number
  of atoms (non-pairs) in a maximally simplified representation of
  S-expression \textit{s-exp} (i.e., the \verb@'()@ terminating proper
  lists should be ignored).

Example log:
\begin{verbatim}
> (count-atoms 'a)
1
> (count-atoms '(a ()))
2
> (count-atoms '(a . ()))
1
> (count-atoms '(a b . ()))
2
> (count-atoms '(a b ()))
3
>
\end{verbatim}  \hfill\textit{20-points}

The key to this question is to decompose the \textit{s-exp}.

\begin{itemize}

\item If \textit{s-exp} is not a pair, then it is an atom
  which should be counted; hence the function should return 1.

\item If \textit{s-exp} is a pair, then the number of atoms
  is the number of pairs in the \verb@car@ plus the number
  of pairs in the \verb@cdr@.  However, when the \verb@cdr@
  is null, then the number of pairs in the \verb@cdr@ should
  be counted as 0.
  
\end{itemize}

This results in the function:

\begin{verbatim}
(define (count-atoms s-exp)
  (if (pair? s-exp)
      (+ (count-atoms (car s-exp))
      (let ([tail (cdr s-exp)])
         (if (null? tail)
             0
             (count-atoms tail))))
      1))
\end{verbatim}

\item Discuss the validity of the following statements:

  \begin{enumerate}

  \item All evaluable Scheme expressions are S-expressions.

  \item All S-expressions are evaluable Scheme expressions.

  \item Assuming that a stack grows towards high memory, then within a
    stack frame for a function, the parameters to the function will be
    located at higher addresses than the local variables of the
    function.

  \item Languages which allow recursive functions \textbf{must} use
    stack allocation for function parameters and local variables.

  \item If a language requires a left-associative binary operator $\oplus$
    to have the same precedence as a right-associative binary operator
    $\otimes$, then the language is ambiguous.

    \end{enumerate}
\hfill\textit{15-points}

  \begin{enumerate}

  \item An evaluable Scheme expression is either an atom or the
    application of a function to zero-or-more scheme expressions, both
    of which can be represented as S-expressions.  Hence the statement
    is \textbf{true}.

  \item Not all S-expressions are evaluable Scheme expressions because
    an evaluable Scheme expression which is a pair requires something
    which represents a function as the first element of the pair.  For
    example, \verb@(1 2)@ is an S-expression but is not an evaluable Scheme
    expression (note that \verb@'(1 2)@ which is the same as
    \verb@(quote (1 2))@ is an evaluable Scheme expression).  Hence the
    statement is \textbf{false}.

  \item The parameters will be pushed onto the stack before the local
    variables.  Since the stack grows towards high memory, the
    parameters will be pushed at lower addresses than the local
    variables.  Hence the statement is \textbf{false}.


  \item Languages which allow recursive functions \textbf{must} use
    stack allocation for function parameters and local variables.

    Languages which allow recursive functions definitely cannot use
    static allocation for function parameters and local
    variables. However, instead of stack allocation, they can use heap
    allocation for function parameters and local variables; this may
    not be particularly efficient but will always work (in fact, this
    may be necessary when a parameter or local variable value escapes
    the function activation).  Hence the statement is \textbf{false}.

  \item When a binary operator is left-associative, then it's
    left-operand can contain a non-parenthesized expression with a
    top-level operator of the same or higher precedence, whereas it's
    right operand can contain a non-parenthesized expression with a
    top-level operator with strictly higher precedence.  A similar
    statement holds for a right-associative operator.

    Consider $x \otimes y \oplus z$.  This expression could represent
    either $x \otimes (y \oplus z)$ or $(x \otimes y) \oplus z$ without
    violating the above associativity constraint.  So the same expression
    can represent multiple syntactic structures and the language is
    ambiguous.  Hence the statement is \textbf{true}.

    \end{enumerate}

\end{enumerate}

\end{document}
