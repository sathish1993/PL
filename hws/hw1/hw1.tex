\documentclass[12pt]{article}
%\usepackage{html}
\usepackage{hyperref}
\title{CS 571\\Homework 1}
%\addtolength{\topmargin}{-2cm}
%\addtolength{\topskip}{-2cm}
%\addtolength{\oddsidemargin}{-2cm}
%\addtolength{\evensidemargin}{-2cm}
%\addtolength{\textheight}{2cm}
%\addtolength{\textwidth}{2cm}
%\addtolength{\footskip}{-1cm}
\date{}
\begin{document}
\maketitle

\begin{flushleft}
\textbf{Due}: Sep 28\hfill\textbf{100 points}\\

\vspace{0.5cm}

\textbf{Important Reminder}: As per the course Academic Honesty
Statement, cheating of any kind will minimally result in receiving an
F letter grade for the entire course.


\end{flushleft}

To be submitted on paper in class.

\begin{enumerate}

\item \href{http://ruby-doc.com/docs/ProgrammingRuby/}{The Ruby
  Programming Language} book describes a ruby integer literal as follows:

  \begin{quote}
    You write integers using an optional leading sign, an optional base indicator (0 for octal, 0x for hex, or 0b for binary), followed by a string of digits in the appropriate base. Underscore characters are ignored in the digit string. 
  \end{quote}

  Note that the above description is somewhat informal; the letters
  mentioned in the above description may be either upper or lower
  case and the underscore characters can only occur \textbf{strictly within}
  the digit string.

  Provide a regular expression which describes ruby integers. \hfill\textit{5-points}

\item The \href{https://docs.oracle.com/javase/specs/jls/se6/html/lexical.html#3.10.2}{Java Language Specification} describes floating point numbers as
  follows:

  \begin{quote}

    A floating-point literal has the following parts: a whole-number
    part, a decimal or hexadecimal point (represented by an ASCII
    period character), a fractional part, an exponent, and a type
    suffix. A floating point number may be written either as a decimal
    value or as a hexadecimal value. For decimal literals, the
    exponent, if present, is indicated by the ASCII letter e or E
    followed by an optionally signed integer. For hexadecimal
    literals, the exponent is always required and is indicated by the
    ASCII letter p or P followed by an optionally signed integer.

    For decimal floating-point literals, at least one digit, in either
    the whole number or the fraction part, and either a decimal point,
    an exponent, or a float type suffix are required. All other parts
    are optional. For hexadecimal floating-point literals, at least
    one digit is required in either the whole number or fraction part,
    the exponent is mandatory, and the float type suffix is optional.

    A floating-point literal is of type float if it is suffixed with
    an ASCII letter F or f; otherwise its type is double and it can
    optionally be suffixed with an ASCII letter D or d.

  \end{quote}

  Note also that a hexadecimal floating point number must start with
  either \verb@0x@ or \verb@0X@ and its significand (i.e. the whole
  number and fraction parts) may contain hexadecimal digits while its
  exponent is restricted to contain only decimal digits.

  Using standard regex syntax, give a regex for Java floating-point
  literals.  Instead of writing a single regex, you may use a \textit
  {regex definition} using a CFG-style syntax with a sequence of named
  regex definitions with later regex definitions referring to earlier
  regex definitions by name (enclosed within angle-brackets).  So
  notation like:
\begin{verbatim}
  <digit>
   : [0-9]
   ;
  <integer>
   : <digit>+
   ;
\end{verbatim}
is acceptable.

Your regex need not make any attempt to restrict the range of the
represented numbers.

\hfill\textit{15-points}

\item As mentioned in the \textbf{Rationale for the Requirements} for
  \href{../../projects/prj1/prj1.html}{Project 1}, the requirements
  do not allow any kind of whitespace characters within constructed
  regex's.
  \begin{enumerate}

  \item How would you modify the requirements to permit whitespace
    to be specified?

  \item How would you change the implementation to permit
    whitespace within the constructed regex's. \hfill\textit{10-points}

  \end{enumerate}

\item As mentioned in the \textbf{Rationale for the Requirements} for
  \href{../../projects/prj1/prj1.html}{Project 1}, the translated
  regex's may have redundant parentheses.  How would you modify your
  parser to ensure that the translation contains only non-redundant
  parentheses.

  For example, the project requires the input
  \verb@((chars(a)) . chars(b))+chars(c)@ be translated to
  \verb@(((([a])[b]))|[c])@ where all the parentheses in the
  output are redundant; it could simply be translated to
  \verb@[a][b]|[c]@.  OTOH,
  \verb@chars(a) . chars(b) + chars(c)@ translates to
  \verb@([a]([b]|[c]))@, where not all the parentheses are redundant;
  without redundant parentheses, it would be \verb@[a]([b]|[c])@.

  Hint: consider providing an additional parameter to each parsing
  function which gives it some idea of the context within which it is
  being called.  \hfill\textit{20-points}

\item A language $\cal L$ consists of strings containing $n$
  \verb@a@'s followed by $n + 4$ \verb@b@'s for $n >= 0$.

\begin{enumerate}
\item Give a CFG for the language $\cal L$.

\item Give an inductive proof that your CFG describes precisely the
  language $\cal L$. \hfill\textit{15-points}

\end{enumerate}

\item Why is a recursive-descent parser for a grammar amenable to
  recursive-descent parsing guaranteed to terminate? \hfill\textit{5-points}

\item The following C function in program \href{./t.c}{t.c}:

\begin{verbatim}
static void
f(int a, double b, int c)
{
  int t1 = a + b;
  int t3 = t1 + c;
  double t2 = b + c;
  int *p1 = &a;
  printf("address of a is %p\n", &a);
  printf("address of b is %p\n", &b);
  printf("address of c is %p\n", &c);
  printf("address of p1 is %p\n", &p1);
  printf("address of t1 is %p\n", &t1);
  printf("address of t2 is %p\n", &t2);
  printf("address of t3 is %p\n", &t3);
}  
\end{verbatim}

prints out the addresses of its arguments and local variables.  On one run,
the output was:

\begin{verbatim}
address of a is 0x7ffcd3fe829c
address of b is 0x7ffcd3fe8290
address of c is 0x7ffcd3fe8298
address of p1 is 0x7ffcd3fe82b0
address of t1 is 0x7ffcd3fe82a0
address of t2 is 0x7ffcd3fe82a8
address of t3 is 0x7ffcd3fe82a4
\end{verbatim}

\begin{enumerate}
\item Show the layout for the stack frame for the above function
  invocation.  Specifically, show the stack frame with the specific
  addresses occupied by the different variables.

\item Explain the above layout in terms of what has been mentioned
  in class about stack-frame layout.

\item Guess at possible addresses where the return address for
  the above function invocation may be stored. \hfill\textit{15-points}

  \end{enumerate}

\item Discuss the validity of the following statements:  
  
\begin{enumerate}
\item Garbage collection was a new technique first used by the
  Java programming language.

\item In a typical programming language, if the name referring to an
  entity goes ``out of scope'' and is no longer accessible, then the
  entity becomes inaccessible and can be destroyed.

\item Global variables cannot be used in multi-threaded programs.

\item A scanner must always discard whitespace and comments.

\item Stack allocation can be used for entities having arbitrary
lifetimes. \hfill\textit{15-points}

    
\end{enumerate}

\end{enumerate}


\end{document}
