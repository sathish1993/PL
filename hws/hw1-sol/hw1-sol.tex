\documentclass[12pt]{article}
%\usepackage{html}
\usepackage{hyperref}
\title{CS 571\\Homework 1 Solution}
%\addtolength{\topmargin}{-2cm}
%\addtolength{\topskip}{-2cm}
%\addtolength{\oddsidemargin}{-2cm}
%\addtolength{\evensidemargin}{-2cm}
%\addtolength{\textheight}{2cm}
%\addtolength{\textwidth}{2cm}
%\addtolength{\footskip}{-1cm}
\date{}
\begin{document}
\maketitle

\begin{flushleft}
\textbf{Due}: Sep 28\hfill\textbf{100 points}\\

\vspace{0.5cm}

\textbf{Important Reminder}: As per the course Academic Honesty
Statement, cheating of any kind will minimally result in receiving an
F letter grade for the entire course.


\end{flushleft}

To be submitted on paper in class.

\begin{enumerate}

\item \href{http://ruby-doc.com/docs/ProgrammingRuby/}{The Ruby
  Programming Language} book describes a ruby integer literal as follows:

  \begin{quote}
    You write integers using an optional leading sign, an optional base indicator (0 for octal, 0x for hex, or 0b for binary), followed by a string of digits in the appropriate base. Underscore characters are ignored in the digit string. 
  \end{quote}

  Note that the above description is somewhat informal; the letters
  mentioned in the above description may be either upper or lower
  case and the underscore characters can only occur \textbf{strictly within}
  the digit string.

  Provide a regular expression which describes ruby integers. \hfill\textit{5-points}

Whitespace and \verb@#@-comments added in regex for readability:
  
\begin{verbatim}
   [-+]?(                                      #optional sign
   0[0-7_]*[0-7] |                             #octal
   0[xX][0-9a-fA-F][0-9a-fA-F_]*[0-9a-fA-F] |  #hex
   0[bB][01][01_]*[01] |                       #binary
   [1-9][0-9_]*[0-9]                           #decimal
   )                          
\end{verbatim}

\item The \href{https://docs.oracle.com/javase/specs/jls/se6/html/lexical.html#3.10.2}{Java Language Specification} describes floating point numbers as
  follows:

  \begin{quote}

    A floating-point literal has the following parts: a whole-number
    part, a decimal or hexadecimal point (represented by an ASCII
    period character), a fractional part, an exponent, and a type
    suffix. A floating point number may be written either as a decimal
    value or as a hexadecimal value. For decimal literals, the
    exponent, if present, is indicated by the ASCII letter e or E
    followed by an optionally signed integer. For hexadecimal
    literals, the exponent is always required and is indicated by the
    ASCII letter p or P followed by an optionally signed integer.

    For decimal floating-point literals, at least one digit, in either
    the whole number or the fraction part, and either a decimal point,
    an exponent, or a float type suffix are required. All other parts
    are optional. For hexadecimal floating-point literals, at least
    one digit is required in either the whole number or fraction part,
    the exponent is mandatory, and the float type suffix is optional.

    A floating-point literal is of type float if it is suffixed with
    an ASCII letter F or f; otherwise its type is double and it can
    optionally be suffixed with an ASCII letter D or d.

  \end{quote}

  Note also that a hexadecimal floating point number must start with
  either \verb@0x@ or \verb@0X@ and its significand (i.e. the whole
  number and fraction parts) may contain hexadecimal digits while its
  exponent is restricted to contain only decimal digits.

  Using standard regex syntax, give a regex for Java floating-point
  literals.  Instead of writing a single regex, you may use a \textit
  {regex definition} using a CFG-style syntax with a sequence of named
  regex definitions with later regex definitions referring to earlier
  regex definitions by name (enclosed within angle-brackets).  So
  notation like:
\begin{verbatim}
  <digit>
   : [0-9]
   ;
  <integer>
   : <digit>+
   ;
\end{verbatim}
is acceptable.

Your regex need not make any attempt to restrict the range of the
represented numbers.

\hfill\textit{15-points}

\begin{verbatim}
<dec>
  : [0-9]
  ;
<dec_exp>
  : [eE][-+]?<dec>+
  ;
<suffix>
  : [fFdD]
  ;
<decimal_float>
  : <dec>+(\.<dec>*)<dec_exp>?<suffix>?   #require dec before point + point 
  | <dec>*(\.<dec>+)<dec_exp>?<suffix>?   #require dec after point + point
  | <dec>+(\.<dec>*)?<dec_exp><suffix>?   #require dec before point + exp
  | <dec>+(\.<dec>*)?<dec_exp>?<suffix>   #require dec before point + suffix
  ;
hex
  : [0-9a-fA-F]
  ;
<hex_exp>
  : [pP][-+]?<dec>+
<hex_float>
  : 0[xX]<hex>+(\.<hex>*)?<hex_exp><suffix>? #require hex before point + exp
  | 0[xX]<hex>*(\.<hex>+)<hex_exp><suffix>?  #require hex after point + exp
  ;
<float>
 : <decimal_float>
 | <hex_float>
 ;
\end{verbatim}

Note the overlapping nature of the alternatives.

\item As mentioned in the \textbf{Rationale for the Requirements} for
  \href{../../projects/prj1/prj1.html}{Project 1}, the requirements
  do not allow any kind of whitespace characters within constructed
  regex's.
  \begin{enumerate}

  \item How would you modify the requirements to permit whitespace
    to be specified?

  \item How would you change the implementation to permit
    whitespace within the constructed regex's. \hfill\textit{10-points}

  \end{enumerate}

  \begin{enumerate}

  \item One obvious change to the requirements would use some kind of
    quotation mechanism.  Specifically, the
    quoting mechanism from C-derived languages could be used: any
    character (including a whitespace character or a \verb@\@
    character following a \verb@\@ character would be regarded
    literally.  Hence a ugly regex which matches a space or a \verb@a@
    could be written as \verb@chars(\ , a)@.  Note also that this
    extension would allow something like \verb@chars(,, a)@ to be
    written as \verb@chars(\,, a)@ which is arguably more readable.

  \item The change required in the implementation would merely require
    a change in the scanner.  Specifically, the code in the scanner
    which returns a \verb@CHAR@ token-kind would be changed: if the
    character seen is a \verb@\@, then that character would be skipped
    and the next character is returned as the value of the \verb@CHAR@
    lexeme.

  \end{enumerate}
  
\item As mentioned in the \textbf{Rationale for the Requirements} for
  \href{../../projects/prj1/prj1.html}{Project 1}, the translated
  regex's may have redundant parentheses.  How would you modify your
  parser to ensure that the translation contains only non-redundant
  parentheses.

  For example, the project requires the input
  \verb@((chars(a)) . chars(b))+chars(c)@ be translated to
  \verb@(((([a])[b]))|[c])@ where all the parentheses in the
  output are redundant; it could simply be translated to
  \verb@[a][b]|[c]@.  OTOH,
  \verb@chars(a) . chars(b) + chars(c)@ translates to
  \verb@([a]([b]|[c]))@, where not all the parentheses are redundant;
  without redundant parentheses, it would be \verb@[a]([b]|[c])@.

  Hint: consider providing an additional parameter to each parsing
  function which gives it some idea of the context within which it is
  being called.  \hfill\textit{20-points}

  Each parsing function knows what operator (if any) it encounters.
  It should parenthesize it's \textbf{sub}-expressions iff the
  top-level operator of that sub-expression has lower precedence
  than the operator seen in the parsing function.

  We associate a level with each type of regex: 0 for character
  classes, 1 for closures and 2 for concatenations and 3 for
  alternations.  Parsing functions have an additional parameter giving
  the level of the parent operator.  Additionally, each parsing
  function return the level of its top-level operator (in addition to
  the translation).

  \begin{itemize}

  \item Character classes never need to be parenthesized.  0 should
    be the level returned for character classes.
    

  \item A closure regex $A$\verb@*@ needs to be parenthesized
    as \verb@(@$A$\verb@)*@ iff the incoming level of $A$ is > 1.  1 should
    be the level returned for a closure regex's.

  \item The $A$ in a concatenation regex $AB$ should be parenthesized
    iff the incoming level for $A$ is $> 2$.  The $B$ in a
    concatenation regex $AB$ should be parenthesized iff the incoming
    level for $B$ is $>= 2$. The difference in treatment for $A$ and
    $B$ is because concatenation is left-associative and we would like
    to have the translation retain the associativitiy specified in the
    input.  2 should be the level returned for an alternation regex's.

  \item The $A$ in a alternation regex $A|B$ need never be parenthesized.
    The $B$ in a
    alternation regex $A|B$ should be parenthesized iff the incoming
    level for $B$ is 3.  The difference in treatment for $A$ and
    $B$ is because alternation is left-associative and we would like
    to have the translation retain the associativity specified in the
    input.  3 should be the level returned for an alternation regex's.

  \item A regex which is parenthesized in the input will not be
    parenthesized in the output unless needed to by its context.
    Hence its return level is simply its incoming level.
    


  \end{itemize}

  
Though no code is required for this question, a Ruby parser for ugly regex's which produces translations without redundant parentheses is given in \href{./min\_aren\_parser.rb}{min\_paren\_parser.rb}.  The relevant portions are shown below:
  

\begin{verbatim}
  #Return value of all parsing functions is a 2-element list
  #containing the translated string and the level.
  def uglyRegexp(level) 
    term_ret = term(level)
    return regexpRest(*term_ret)
  end

  def regexpRest(t, level)
    if check(:CHAR, ".") 
      match(:CHAR, ".")
      t1 = term(1)
      return regexpRest("#{paren(2, t, level)}#{paren(1, *t1)}", 2)
    else 
      return [t, level]
     end
  end

  def term(level)
    f = factor(level)
    return termRest(*f)
  end

  def termRest(f, level) 
    if check(:CHAR, "+")
      match(:CHAR, "+")
      f1 = factor(2)
      return termRest("#{paren(3, f, level)}|#{paren(2, *f1)}", 3)
    else 
      return [f, level]
    end
  end

  def factor(level)
    if check(:CHAR, "(") 
      match(:CHAR, "(")
      e = uglyRegexp(level)
      match(:CHAR, ")")
      return e
    elsif check(:CHAR, "*")
      match(:CHAR, "*")
      f = factor(1)
      return [f[0], 1]
    else 
      match(:CHARS)
      match(:CHAR, "(")
      s = @lookahead.lexeme
      match(:CHAR)
      chars = chars(quote(s))
      match(:CHAR, ")")
      return [ "[#{chars}]", 0]
    end
  end

  def chars(s) 
    if check(:CHAR, ",")
      match(:CHAR, ",")
      s1 = @lookahead.lexeme
      match(:CHAR)
      return chars("#{s}#{quote(s1)}")
    else 
      return s
    end
  end

  #return parenthesized exp iff exp_level > level.
  def paren(level, exp, exp_level)
    (exp_level > level) ? "(#{exp})" : exp
  end  
\end{verbatim}

Note that token kinds are denoted using Ruby symbols which are
indicated using a leading \verb@:@.

Since all the parsing functions (except \verb@chars()@ needs to return
2 values (the translation and the level), the return value for the
parsing functions is a 2-element list containing the translation and
the level within square brackets.  To avoid having to index the list
within parsing functions, the 2 values are passed to the parsing
functions as separate arguments.  This is achieved using Ruby's prefix
\verb@*@ \textit{splat} operator; i.e. for list \verb@v@,
\verb@fn(v[0], v[1])@ is written as \verb@fn(*v)@.
    

\item A language $\cal L$ consists of strings containing $n$
  \verb@a@'s followed by $n + 4$ \verb@b@'s for $n >= 0$.

\begin{enumerate}
\item Give a CFG for the language $\cal L$.

\item Give an inductive proof that your CFG describes precisely the
  language $\cal L$. \hfill\textit{15-points}

\end{enumerate}

\begin{enumerate}

\item The grammar using terminals \verb@'a'@, \verb@'b'@ and
  non-terminal \verb@S@ is shown below:
\begin{verbatim}
    S
    : 'b' 'b' 'b' 'b'    //(1)
    | 'a' S 'b'          //(2)
    ;
\end{verbatim}
    
\item It is
  clear from the description of the language that strings in the
  language will have length $2 \times k + 4$ for $k >= 0$.  The proof
  will be by induction over $k$.

  Base case: For $k = 0$, the language consists of \verb@b b b b@.
  This is described by rule (1).

  Assume as induction hypothesis that $S$ generates sentences with $2
  \times k + 4$; i.e. $S$ is of the form
  $\mbox{\texttt{a}}^k\mbox{\texttt{b}}^{k+4}$. Hence rule (2) generates
  $\mbox{\texttt{a}}\mbox{\texttt{a}}^k\mbox{\texttt{b}}^{k+4}\mbox{\texttt{b}}$
  which is $\mbox{\texttt{a}}^{k+1}\mbox{\texttt{b}}^{(k+1)+4}$ which is the
  language for $k + 1$.  Hence the grammar generates the correct language.

\end{enumerate}
\item Why is a recursive-descent parser for a grammar amenable to
  recursive-descent parsing guaranteed to terminate? \hfill\textit{5-points}

  The only way a recursive-descent parser will fail to terminate is if
  there is an infinite loop involving direct or indirect recursion.
  If a grammar is amenable to recursive-descent parsing then before
  the parser makes any recursive calls it must have \verb@match@'d
  one-or-more tokens, consuming those tokens.  Since the size of the
  token-stream is bounded, the parser must terminate.

\item The following C function in program \href{./t.c}{t.c}:

\begin{verbatim}
static void
f(int a, double b, int c)
{
  int t1 = a + b;
  int t3 = t1 + c;
  double t2 = b + c;
  int *p1 = &a;
  printf("address of a is %p\n", &a);
  printf("address of b is %p\n", &b);
  printf("address of c is %p\n", &c);
  printf("address of p1 is %p\n", &p1);
  printf("address of t1 is %p\n", &t1);
  printf("address of t2 is %p\n", &t2);
  printf("address of t3 is %p\n", &t3);
}  
\end{verbatim}

prints out the addresses of its arguments and local variables.  On one run,
the output was:

\begin{verbatim}
address of a is 0x7ffcd3fe829c
address of b is 0x7ffcd3fe8290
address of c is 0x7ffcd3fe8298
address of p1 is 0x7ffcd3fe82b0
address of t1 is 0x7ffcd3fe82a0
address of t2 is 0x7ffcd3fe82a8
address of t3 is 0x7ffcd3fe82a4
\end{verbatim}

\begin{enumerate}
\item Show the layout for the stack frame for the above function
  invocation.  Specifically, show the stack frame with the specific
  addresses occupied by the different variables.

\item Explain the above layout in terms of what has been mentioned
  in class about stack-frame layout.

\item Guess at possible addresses where the return address for
  the above function invocation may be stored. \hfill\textit{15-points}

  \end{enumerate}

\begin{enumerate}
\item The layout is shown below:

\begin{verbatim}
                     +----------------+
      0x7ffcd3fe8290 |       b        | 
      0x7ffcd3fe8294 |                |
                     +----------------+
      0x7ffcd3fe8298 |       c        |
                     +----------------+
      0x7ffcd3fe829c |       a        |
                     +----------------+
      0x7ffcd3fe82a0 |       t1       |
                     +----------------+
      0x7ffcd3fe82a4 |       t3       |
                     +----------------+
      0x7ffcd3fe82a8 |       t2       |
                     +----------------+
      0x7ffcd3fe82b0 |       p1       |
      0x7ffcd3fe82b4 |                |
                     +----------------+
\end{verbatim}

Note that there is nothing in the problem indicating the size of the
variables; however, from the addresses it seems reasonable to assume
that the code was run on a 64-bit machine with integers having size
4 and doubles and pointers having size 8.

\item This problem did not work out as intended. \verb@:-(@

  Assuming that the stack grows towards low memory, based on what was
  mentioned in class the arguments should have been located at higher
  addresses and the local variables at lower addresses with some
  seemingly unused space between the argument area and the local
  variables.  That is not the case in the above trace, as the compiler
  seems to have optimized the stack layout.

\item If the problem had worked out as intended, the return
  address would have been stored in the space between the
  area uses for the arguments and the area used for the local
  variables.  However, with the given addresses there does not
  appear to be any place within the stack frame for the return
  address; presumably it was stored within a register.

  \end{enumerate}

\item Discuss the validity of the following statements:  
  
\begin{enumerate}
\item Garbage collection was a new technique first used by the
  Java programming language.
\item In a typical programming language, if the name referring to an
  entity goes ``out of scope'' and is no longer accessible, then the
  entity becomes inaccessible and can be destroyed.

\item Global variables cannot be used in multi-threaded programs.

\item A scanner must always discard whitespace and comments.

\item Stack allocation can be used for entities having arbitrary
  lifetimes. \hfill\textit{15-points}

\end{enumerate}

\begin{enumerate}
\item Garbage collection has been around since one of the earliest
  languages: Lisp.  Hence the statement is \textbf{false}.
  
\item This is not necessarily true. For example, when a function
  uses a reference parameter returns, the reference parameter
  goes out of scope but the entity referred to by that parameter
  may still be accessible in the caller and cannot be destroyed.
  Hence the statement is \textbf{false}.

\item Global variables can be used in multi-threaded programs.  They
  can definitely be used if they are read-only variables.  Even if
  they are read-write variables they can be used with locks to ensure
  no inconsistencies when used by multiple threads.  Hence the
  statement is \textbf{false}.


\item This statement would be true for scanner's used for compilers.
  However this would be unacceptable behavior for a scanner used for a
  syntax-directed editor.  Hence the statement is \textbf{false}.


\item Stack allocation can only be used for entities with nested
  lifetimes typically corresponding to function activations.  Hence
  the statement is \textbf{false}.

\end{enumerate}

\end{enumerate}


\end{document}
